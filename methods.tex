\section{Methods} \label{sec:methods}
\subsection{Experimental}

For the present experiment we used the newly operational microjet setup that was specifically designed for the HAXPES station of the GALAXIES beamline \citep{ceolin13:188,rueff15:175}. The aqueous potassium chloride solution was prepared by mixing >99\% KCl salt with deionized water. Filtering and degazing procedures were systematically performed before injecting the solution. The spectrometer resolution of about 0.6\,eV was achieved with the 500\,eV pass energy and 0.5\,mm slits. The photon energy resolution achieved at 2.8\,keV and 3.6\,keV was about 0.3\,eV and 0.4\,eV, respectively. The experimental 2D maps representing the evolution of the KLL Auger spectra in the vicinity of the \cli and \ki~K-edges, as a function of the photon energy, are shown in Figs.\ \ref{fg:2dmap_k} and \ref{fg:2dmap_cl}, respectively. The aqueous \ki~and \cli~1s ionization potentials were measured at h$\nu$=5keV and calibrated on the liquid contribution of the O1s XPS spectrum \cite{winter06:1176}. The maps were also calibrated using the O1s photoelectron line of liquid water but at photon energies close to the potassium and chloride 1s ionization thresholds.


\subsection{{\bf{\it Ab initio}} calculations}

The theoretical X-ray absorption spectra were computed for the hexa-coordinated clusters of both ions, \ki(H$_2$O)$_6$ and \cli(H$_2$O)$_6$, which can be considered as representatives of the complete first solvation shell \citep{Ohtaki93:1157,soper06:180,ma14:1006}. The two structures shown in Fig.\ \ref{fg:xas_kcl} were optimized at the DFT level of theory using the B3LYP functional and the 6-311++G(2d,2p) basis set \citep{Krishnan80:650,Blaudeau97:5016}. The geometry optimization was performed with the Gaussian 09 package \citep{g09}.


The energies and transition moments of the core excited states of the bare ions and microsolvated clusters were computed with the Algebraic Diagrammatic Construction method for the polarization propagator \citep{sch82:2395} within the core-valence separation approximation \citep{bar85:867,ced80:206,ced81:1038} (CVS-ADC(2)x) as implemented in the Q-Chem package \citep{Wenzel14:1900,Wenzel14:4583,Wormit14:774,QChem2015}. In the case of \cli~the 6-311++G(3df,3pd) basis set \citep{Krishnan80:650,McLean80:5639} (excluding f functions) was used for all atoms, whereas in the case of \ki~we used the 6-311+G(2d,p) basis set \citep{Krishnan80:650,Blaudeau97:5016} on all atoms, and two additional sets of s, p and d diffuse functions were added on K. In our calculations the core space comprises the 1s orbital of K$^{+}$ or \cli, whereas the remaining occupied orbitals are included in the valence space. We analyzed the core excited states by expanding the natural orbitals occupied by the excited electron (singly occupied natural orbitals, SONOs) of the microsolvated clusters in the basis of SONOs of the bare ions as in \citep{miteva16:16671}. 

The final states following KLL resonant Auger decay of K$^{+}$(H$_2$O)$_6$ and \cli(H$_2$O)$_6$ were computed at the Configuration Interaction Singles (CIS) level using the Graphical Unitary Group Approach (GUGA) as implemented in the GAMESS-US package \citep{GUGA_PhysScr_21,GUGA_JCP_70,GUS}. In order to account for the relaxation effects upon core ionization, we used a restricted open-shell Hartree-Fock reference wave function with a hole in the 2s orbital of both \ki~and \cli.  We used the 6-311++G(2d,2p) basis set \citep{Krishnan80:650,McLean80:5639,Blaudeau97:5016} on all atoms augmented with two sets of s, p, d diffuse functions in the case of \ki, and three sets of s, p, d diffuse functions in the case of \cli. The active space comprises the 2s and 2p orbitals of K/Cl with occupancy fixed to 6 and all virtual orbitals with occupancy fixed to 1. The remaining doubly occupied orbitals were frozen in the calculation. \citep{mosnier16:061401}