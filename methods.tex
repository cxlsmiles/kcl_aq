%\section{Methods} \label{sec:methods}
%\subsection{Experimental}

For the present experiment we used the newly operational microjet setup that was specifically designed for the HAXPES station of the GALAXIES beamline \citep{ceolin13:188,rueff15:175}. {\color{red}CAN WE MAKE THIS SHORTER? A differentially-pumped tube in which the microjet head is inserted, is mounted on a 3-axes motorized manipulator in front of the spectrometer lens. Two holes of 2\,mm diameter allow the photons to go in and out. At the end of the tube and in front of the lens, a 500\,$\mu$m diameter hole skimmer allows the electrons created at the interaction point to go in the direction of the spectrometer. The microjet head is mostly composed of a 30\,$\mu$m diameter vertical glass capillary facing a temperature-controlled catcher in CuBe having a 300\,$\mu$m hole, and a camera. Piezo motors allow their precise alignment relative to each other and to the photon beam. The catcher is placed at a distance of about 5\,mm from the capillary and is permanently pumped in order to extract the liquid. For the present experiment, a 0.5M KCl aqueous solution is injected in the capillary by a high performance liquid chromatography (HPLC) pump with a constant flux of 1.6\,ml/min. The alignment of the setup is performed on the KCl aqueous solution by measuring the water O1s x-ray photoelectron peak intensity and by optimizing the liquid vs gas phase ratio. The pressure in the main chamber is kept below the 10$^{-5}$\,mbar range whereas it is kept at about 10$^{-4}$\,mbar in the differentially-pumped tube when the HPLC pump is ON. Our equipment is an updated version of the equipment used in Ref.\ \cite{faubel88:269}.} The aqueous potassium chloride solution was prepared by mixing >99\% KCl salt with deionized water. Filtering and degazing procedures were systematically performed before injecting the solution. The spectrometer resolution of about 0.6\,eV was achieved with the 500\,eV pass energy and 0.5\,mm slits. The photon energy resolution achieved at 2.8\,keV and 3.6\,keV was about 0.3\,eV and 0.4\,eV, respectively. The experimental 2D maps representing the evolution of the KLL Auger spectra in the vicinity of the \cli and \ki~K-edges, as a function of the photon energy, are shown in Figs.\ \ref{fg:2dmap_k} and \ref{fg:2dmap_cl}, respectively. The aqueous \ki~and \cli~1s ionization potentials were measured at h$\nu = 5$\,keV and calibrated on the liquid contribution of the O1s XPS spectrum \cite{winter06:1176}. The maps were also calibrated using the O1s photoelectron line of liquid water but at photon energies close to the potassium and chloride 1s ionization thresholds.


%\subsection{{\bf{\it Ab initio}} calculations}

The theoretical X-ray absorption spectra were computed for the hexa-coordinated clusters of both ions, \ki(H$_2$O)$_6$ and \cli(H$_2$O)$_6$, which can be considered as representatives of the complete first solvation shell of the two ions \citep{Ohtaki93:1157,soper06:180,ma14:1006}. The two structures shown in Fig.\ \ref{fg:xas_kcl} were optimized at the DFT level of theory using the B3LYP functional and the 6-311++G(2d,2p) basis set \citep{Krishnan80:650,Blaudeau97:5016}. The geometry optimization was performed with the Gaussian 09 package \citep{g09}. In order to obtain a realistic structure for \ki(H$_2$O)$_6$ corresponding to the bulk solution, we carried out constrained geometry optimization by fixing the K-O distance to 2.840~\AA~and increasing the angle $\theta$ between the K-O bond and the $C_3$ axis to 55$^{\circ}$ using the equilibrium gas-phase geometry belonging to the D$_3$ point group as a starting point \citep{lee99:3995,lee02:5509}.
%  and then increasing the angle $\theta$ between the K-O bond and the $C_3$ axis to 55$^{\circ}$. This angle was chosen such that the O-K-O angles are around the maxima of the angular distributions obtained from quantum mechanics/molecular mechanics simulations in Ref.\ \citep{ma14:1006}. Moreover, we fixed the K-O distance to 2.840~\AA, such that it corresponds to the distances from other theoretical and experimental works \citep{Ohtaki93:1157,soper06:180,ma14:1006}.
%\citep{ge13:13169,gora00:7,Ohtaki93:1157,soper06:180,ma14:1006}.
%Moreover, we fixed the K-O and Cl-O distances to 2.840~\AA~and 3.140~\AA, respectively, such that they correspond to the distances from other theoretical and experimental works \citep{ge13:13169,gora00:7,Ohtaki93:1157,soper06:180,ma14:1006}.

The energies and transition moments of the core excited states of the bare ions and microsolvated clusters were computed with the algebraic diagrammatic construction method for the polarization propagator \citep{sch82:2395} within the core-valence separation approximation \citep{bar85:867,ced80:206,ced81:1038} (CVS-ADC(2)x) as implemented in the Q-Chem package \citep{Wenzel14:1900,Wenzel14:4583,Wormit14:774,QChem2015}. In the case of \cli~the 6-311++G(3df,3pd) basis set \citep{Krishnan80:650,McLean80:5639} (excluding f functions) was used on all atoms, whereas in the case of \ki~we used the 6-311+G(2d,p) basis set \citep{Krishnan80:650,Blaudeau97:5016} on all atoms, and two additional sets of s, p and d diffuse functions were added on K. %The use of a smaller basis set in the case of K is due to the higher number of atomic orbitals compared to the case of Cl, and therefore, prohibitively high cost of the CVS-ADC(2)x computation. 
In our calculations the core space comprises the 1s orbital of K$^{+}$ or \cli, whereas the remaining occupied orbitals are included in the valence space. %For the calculations of the XAS spectra we used the C$_2$ point group in the case of K$^{+}$(H$_2$O)$_6$ and \cli(H$_2$O)$_6$. 
To account for the experimental resolution and the lifetime broadening due to the Auger decay of the core excited states, we convolved the theoretical spectra with a Gaussian profile and a Lorentzian function of FWHM 0.74\,eV and 0.62\,eV in the case of \ki~and \cli, respectively \citep{Krause79:329}. In order to understand the mixing of the core excited states in the ligand field created by the surrounding water molecules, we analyzed the core excited states of the hexa-coordinated clusters by expanding the natural orbitals occupied by the excited electron (singly occupied natural orbitals, SONOs) in the basis of SONOs of the bare K$^{+}$ or \cli~ions, as described in Ref.\ \citep{miteva16:16671}. 


The final states following KLL resonant Auger decay of K$^{+}$(H$_2$O)$_6$ and \cli(H$_2$O)$_6$ were computed at the Configuration Interaction Singles (CIS) level using the Graphical Unitary Group Approach (GUGA) as implemented in the GAMESS-US package \citep{GUGA_PhysScr_21,GUGA_JCP_70,GUS}. In order to account for the relaxation effects upon core ionization, we employed a restricted open-shell Hartree-Fock reference wave function with a hole in the 2s orbital of both \ki~and \cli.  We used the 6-311++G(2d,2p) basis set \citep{Krishnan80:650,McLean80:5639,Blaudeau97:5016} on all atoms. Additionally, the basis set was augmented with two sets of s, p, d diffuse functions in the case of \ki, and three sets of s, p, d diffuse functions in the case of \cli. %The larger basis set employed in the case of Cl was necessary in order to ensure the convergence of the excited states. 
The active space comprises the 2s and 2p orbitals of K/Cl with occupancy fixed to 6 and all virtual orbitals with occupancy fixed to 1. The remaining doubly occupied orbitals were frozen in the calculation. \citep{mosnier16:061401}