\section{Conclusion}\label{sec:concl}

Using a combination of x-ray absorption and Auger electron spectroscopy in the hard x-ray regime, in this work we study the electronic structure of aqueous solution of KCl at the K-edges of both K and Cl. The Auger electron spectra of both ions as a function of photon energy exhibit features of normal as well as resonant Auger processes. To interpret the resonant Auger features in the experimental spectrum, we performed {\it ab initio} calculations on microsolvated clusters of \ki~and \cli. Our calculations show that the energy ordering of the 3d and 4p virtual orbitals of \cli~is inverted compared to \ki, and also that the energy splitting between the bright 1s$\,\rightarrow\,$4p and dark 1s$\,\rightarrow\,$3d core excited states is larger in the chlorine case. Thus, the energetic proximity of the 3d and 4p orbitals in the bare \ki~ion results in the dipole forbidden 1s$\,\rightarrow\,$3d state acquiring intensity in a solution as a result of mixing with the dipole allowed 1s$\,\rightarrow\,$4p excitation. The spectator Auger decay of this state produces an additional dispersive feature which is manifest as a separate island in the Auger electron spectrum at high kinetic energies. In the case of \cli~the two core excited states do not interact, and therefore, only fingerprints of the population and Auger decay of the dipole allowed 1s$\,\rightarrow\,$4p state are observed in the spectrum. Moreover, using the core-hole clock method we estimated the time of delocalization of the core excited electron at the pre-edge region of both ions. Whereas in \ki$_{\text{aq}}$ the delocalization of the core excited electron is slower than the resonant Auger process, in the case of \cli$_{\text{aq}}$ the two processes occur on a comparable timescale.


{\bf\color{red}A nice concluding ``wow'' sentence}
%Our work shows that the combination of the two techniques is a very sensitive probe of the electronic structure of solvated ions, and it1
%Finally, we would like to stress the importance of the combination of the two experimental techniques for the study of aqueous solutions of ions.