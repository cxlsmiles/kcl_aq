%\section{Conclusion}\label{sec:concl}

In summary, we studied the electronic structure of aqueous solution of KCl at the K-edges of both K and Cl using a combination of x-ray absorption and Auger electron spectroscopy in the tender x-ray regime, and {\it ab initio} calculations. The Auger electron spectra of both ions as a function of photon energy exhibit features of normal as well as resonant Auger processes. The latter process proceeds differently for aqueous K$^{+}$ and Cl$^{-}$ due to the population of the dipole forbidden K$^{+}$ 1s$^{-1}$3d state in a solution. The spectator Auger decay of this state produces an additional dispersive feature which is manifested as a separate peak in the Auger electron spectrum. In the case of \cli~only fingerprints of the population and Auger decay of the dipole allowed 1s$^{-1}$4p excitation are observed in the spectrum.


%Our work shows that the combination of x-ray absorption and resonant Auger spectroscopy is a sensitive probe of the electronic structure of solvated ions. 
The reported results are an important first step in the study of the chains of relaxation steps triggered by photoabsorption in the tender x-ray regime. The Auger processes considered here are inevitably followed by multiple intra- and interatomic electronic decays, such as interatomic Coulombic decay (ICD) and electron-transfer mediated decay (ETMD). As a result of the latter processes, genotoxic free radicals and slow electrons are formed in the vicinity of the metal center. The magnitude of the damage inflicted upon the environment and the energies of the emitted electrons depend on the initial Auger step, and can therefore be controlled by tuning the energy of the radiation. Consequently, the results of this work can have implications in understanding radiation chemistry and radiation damage in biologically relevant systems.