X-ray absorption and (resonant) Auger electron spectroscopies are powerful tools to probe the electronic structure and immediate surroundings of ions in a solution. In this work we combined these two methods to study the electronic structure and decay processes of aqueous KCl at the \ki~and \cli  K-edges.
%Auger electron spectroscopy is a powerful tool to probe the electronic structure and immediate surroundings of ions in a solution. In this work we use a combination of x-ray absorption and Auger electron spectroscopies to study the electronic structure and decay of aqueous KCl at the K-edges of \ki~and \cli. 
Although the two ions are isoelectronic, their Auger spectra as a function of the photon energy exhibit notably different features. To explain these differences, we carried out {\it ab initio} calculations on both the core-excited states and the final Auger states of \ki, \cli~and on microsolvated clusters at given geometries. \denis{Our calculations highlight the important role of the 3d orbitals, whose energy position compared with the 4p orbitals is inverted in \ki~with respect to \cli}. 
%
%The reverse orbital order in the two ions not only influences the x-ray absorption spectra, but also impacts the course of the subsequent resonant Auger processes.
%
The reverse orbital order in the two ions is reflected in both the ordering of the core-excited states and in the final states populated by resonant Auger decay. Furthermore, the energetic proximity of the 3d and 4p virtual states in the bare \ki~ion leads to their mixing in the presence of the solvent, and to the population of the dipole forbidden 1s$\,\rightarrow\,$3d state upon K-shell excitation in an aqueous solution. The resonant Auger decay of this state results in a separate feature in the Auger electron spectrum of \ki~which is absent in the spectrum of \cli.