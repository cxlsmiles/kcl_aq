X-ray absorption and Auger electron spectroscopies are demonstrated to be powerful tools to unravel the electronic structure of solvated ions. In this work for the first time we use a combination of these methods in the tender x-ray regime. This allowed us to explore several hitherto unaddressed electronic transitions and to probe environmental effects, specifically in the bulk of the solution. In the considered exemplary aqueous KCl solution the solvated isoelectronic \ki~and \cli~ions exhibit notably different Auger electron spectra as a function of the photon energy. The differences appear due to dipole forbidden transitions in aqueous \ki~whose occurrence, according to the performed {\it ab initio} calculations, becomes possible only in the presence of solvent water molecules.

%X-ray absorption (XAS) and Auger electron spectroscopies (AES) are powerful tools to probe the electronic structure and immediate surroundings of ions in solution. In this work we use a combination of these methods to study the electronic structure and decay of an exemplary aqueous KCl at the K-edges of \ki~and \cli. Although the two ions are isoelectronic, their Auger electron spectra as a function of the photon energy exhibit notably different features resulting from the excitation and Auger decay of dipole forbidden states in aqueous \ki. The results of this work represent a pioneering study of the decay processes initiated by photoabsorption in the tender x-ray regime close to threshold in liquids. Using tender x-rays we can probe the bulk of the solution, whereas the combination of XAS and resonant Auger spectroscopy allows us to unambiguously disentangle the electronic structure of the solvated ions.

%. These features result from the excitation and Auger decay of dipole forbidden states in aqueous \ki, which become dipole allowed due to mixing with the closely lying bright states in the presence of the solvent. The results of this work confirm the sensitivity of core level spectroscopies to the electronic structure of solvated species and represent a pioneering study of the decay processes initiated by photoabsorption in the tender x-ray regime close to threshold in liquids.