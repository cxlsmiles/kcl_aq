X-ray absorption and Auger electron spectroscopies are powerful tools to probe the electronic structure and immediate surroundings of ions in solution. Using these techniques, we study the electronic structure and decay of aqueous KCl at the K-edges of \ki~and \cli. Although the two ions are isoelectronic, their Auger electron spectra exhibit notably different features. This is due to the energetic proximity of the 1s$^{-1}$3d and 1s$^{-1}$4p core excited states in the bare \ki~ion leading to their mixing in the presence of the solvent. As a result, the dipole forbidden 1s$^{-1}$3d state is populated upon K-shell excitation of aqueous K$^{+}$, and its resonant Auger decay of this state results in a separate feature in the Auger electron spectrum of \ki~which is absent in the spectrum of \cli. The results of this work represent a pioneering study of the decay processes initiated by photoabsorption in the tender x-ray regime close to threshold in liquids.


%X-ray absorption and Auger electron spectroscopies are powerful tools to probe the electronic structure and immediate surroundings of ions in solution. In this work we use a combination of these methods together with {\it ab initio} calculations to study the electronic structure and decay in aqueous KCl at the K-edges of \ki~and \cli. Although the two ions are isoelectronic, their Auger electron spectra exhibit notably different features. To explain these differences, we carried out {\it ab initio} calculations of both the core excited states and the final Auger states of bare \ki, \cli~and their microsolvated clusters. This is due to the fact that the energetic order of the 1s$^{-1}$3d and 1s$^{-1}$4p core excited states is inverted in \ki~with respect to \cli, and moreover the two states lie energetically close in the bare \ki~ion leading to their mixing in the presence of the solvent. As a result, the dipole forbidden 1s$^{-1}$3d state is populated upon K-shell excitation of aqueous K$^{+}$. The resonant Auger decay of this state results in a separate feature in the Auger electron spectrum of \ki~which is absent in the spectrum of \cli. The results of this work represent a pioneering study of the decay processes initiated by photoabsorption in the tender x-ray regime close to threshold in liquids and are thus of importance in unveiling the mechanisms of radiation damage in biologically relevant systems.
%Denis: line 2, I replaced spectroscopies by methods