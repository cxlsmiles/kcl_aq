X-ray absorption and Auger electron spectroscopies are demonstrated to be powerful tools to unravel the electronic structure of solvated ions. In this work for the first time we use a combination of these methods in the tender x-ray regime. This allowed us to address electronic transitions from deep core levels and to probe environmental effects, specifically in the bulk of the solution since the created energetic Auger electrons possess large mean free paths. In the considered exemplary aqueous KCl solution the solvated isoelectronic \ki~and \cli~ions exhibit notably different Auger electron spectra as a function of the photon energy. The differences appear due to dipole-forbidden transitions in aqueous \ki~whose occurrence, according to the performed {\it ab initio} calculations, becomes possible only in the presence of solvent water molecules.